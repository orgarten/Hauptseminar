\section{Aufgabenstellung}
Es ist ein Programm zu entwerfen, welches Audiodateien einliest und dateiweise die beiden Stereo-Kanäle miteinander korreliert. Es sollen Maßzahlen entworfen und berechnet werden, die wesentliche charakteristische Eigenschaften der Korrelationsfunktion, insbesondere den Anteil dominanter Komponenten und deren Abklingverhalten, widerspiegeln. Dafür sind Audiosignale aufzunehmen, bezüglich der verwendeten Maße zu klassifizieren und entsprechend ihrer Klassifzierung systematisch abzuspeichern.
%%%%%%%%%%%%%%%%%%%%%%%%%%%%%%
\section{Motivation}
Im Jahr 2022 werden 500 Milliarden internetfähige Geräte erwartet, die mit einander kommunizieren sollen. Das führt zu extrem hohen Datenmengen, die in kürzestmöglicher Zeit von A nach B transportiert werden müssen. Große Herausforderungen bestehen darin, dass man sehr kurze Verzögerungszeiten und eine hohe Widerstandsfähigkeit garantieren muss. Idealerweise benötigen die Geräte wenig Energie. Ein Ansatz zur Lösung dieses Problems ist die Netzwerkcodierung.\newline
In bestimmten Szenarien ist eine große Anzahl an Geräten mit Sensorik zu Erfassung der Umgebung mit hohen Anforderungen an die Netzwerkkapazität zur Übertragung der erfassten Daten verbunden. Es stellt sich die Frage, wie viele Sensoren für eine ausreichend genaue Abbildung benötigt werden. Da die Quellen teilweise korrelierte Datenströme erzeugen, lässt sich die zu übertragende Gesamtdatenmenge reduzieren, was durch eine geeignete Kombination von Netzwerkcodierung mit Methoden des Compressed Sensing erreicht werden soll.

%%%%%%%%%%%%%%%%%%%%%%%%%%%%%%
\section{Theoretische Vorbetrachtung}
\subsection{Kreuzkorrelationsfunktion}
Die Basis f"ur die Bemessung der aufgenommenen Audiosignale bildet die sogenannte Kreuzkorrelationsfunktion (KKF). Wie in der Aufgabenstellung schon beschrieben werden an ihr die Bemessungsparameter festgelegt. Aufgrund der verschiedenen Blockl"angen und der Masse an Daten, die korreliert werden sollen muss die Berechnung der KKF effizient und zeitsparend implementiert werden. Im Folgenden Abschnitt wird die KKF kurz theoretisch eingef"uhrt und das das mathematische Konzept erkl"art, auf dem die effiziente Berechnung der KKF beruht.
\\\\
Zuerst haben wir f"ur die KKF eine Funktion genutzt, die zur Berechnung Summen verwendete. Dabei ergab sich, dass die Berechnung zu langsam war. In der nun vorliegenden Octave-Version wird die KKF im Frequenzbereich berechnet.
\subsubsection{Berechnungsvorschrift}
Die KKF ist als aus zwei verschiedenen Funktionen gebildeter Erwartungswert definiert. Hier werden die Formeln allgemein f"ur die Korrelation der Prozesse \textbf{X} und \textbf{Y} angegeben. 

\begin{align}
\psi_{\textbf XY}(t_1, t_2) = E\lbrace\textbf{X}(t_1)\cdot\textbf{Y}(t_2)\rbrace
\end{align}
Reale aufgenommene Audiosignale s(t), die hier mit durch die KKF verrechnet werden, sind in jedem Fall Energiesignale, da sie rein reell sind, einen begrenzten Wertebereich haben und nach einer bestimmten Zeit enden.

\begin{align}
Signalenergie := \int_{-\infty}^{\infty} s^2(t) \,dt < \infty
\end{align}

\noindent Da die aufgenommenen Audiosignale zeit-diskrete Energiesignale sind wird hier auch nur die zeit-diskrete Kreuzkorrelation beschrieben. F"ur zeit-diskrete Energiesignale ergibt sich die folgende Berechnungsvorschrift, wobei x(n) und y(n) Realisierungen der Prozesse \textbf{X} und \textbf{Y} sind.

\begin{align}
\boxed{\psi_{\textbf {XY}}^E(k) = \sum_{n = -\infty}^{\infty} x(n) \cdot y(n+k)}
\end{align}
vgl. [ISV] S. 84 folgende

\subsubsection{Kreuzkorrelation im Frequenzbereich und Faltungssatz}
Am Anfang unserer Arbeit haben wir uns mit der Berechnung der Kreuzkorrelation besch"aftigt. Da die Bildung der Summe der Multiplikation von x(n) und y(n+k) sehr rechenaufw"andig ist, haben wir nach schnelleren M"oglichkeiten gesucht die KKF der beiden Signale zu berechnen. Eine geeignet M"oglichkeit ist die Berechnung der KKF im Frequenzbereich. F"ur die Berechnung der KKF im Frequenzbereich macht man sich die "Ahnlichkeit der KKF zur Faltung und den Faltungssatz zunutze.

\paragraph{KKF als Faltung}
%hier herrscht noch unsicherheit vor, ich habe nur die Formel für den kont. Fall
\textbf{\\\\Zeit-kontinuierlicher Fall:}
\begin{align}
\psi_{\textbf {XY}}^E(\tau) = x(-\tau) * y(\tau)
\end{align}
\textbf{Zeit-diskreter Fall:}
\begin{align}
\psi_{\textbf {XY}}^E(k) = x(-k) * y(k)
\end{align}
[ISV] Formel (2.217) S.89

\paragraph{Faltungssatz in Verbindung mit KKF}

Wir schreiben zuerst die die KKF als Faltung. Danach transformieren wir die Faltung in den Frequenzbereich. Durch geschicktes Erweitern und Substitution findet man einen Ausdruck, um die KKF im Frequenzbereich zu berechnen.

\begin{align}
\psi(t) \  \; &= x(-t) * y(t) = \int_{-\infty}^{\infty} x(-\tau) \cdot y(t-\tau)\,d\tau\\\\
\underline \Psi(\omega) &=\int_{-\infty}^{\infty} x(-t) * y(t) \cdot e^{-j\omega t} \,dt\\&=\int_{-\infty}^{\infty} \left[ \int_{-\infty}^{\infty} x(-\tau) \cdot y(t-\tau)\,d\tau \right] \cdot e^{-j\omega (t - \tau + \tau)} \,dt\\&=\int_{-\infty}^{\infty} x(-\tau) \cdot e^{j\omega(-\tau)} \underbrace{ \left[ \int_{-\infty}^{\infty} y(t-\tau) e^{-j\omega (t - \tau)} \,dt \right]}_* \,d\tau
\end{align}
Durch Substitution von $(t-\tau)$ durch $t'$ ergibt sich * zur Fourier-transformierten $ \underline{Y}(\omega)$ von y(t). Der restliche Ausdruck wird durch das positive Vorzeichen in der e-Funktion zur komplex konjugierten Transformierten $ \underline{X}^*(\omega)$ der Funktion x(t).
\\\\
Die KKF l"asst sich im Frequenzbereich also als

\begin{align}
\boxed{\underline \Psi(\omega) = \underline{X}^*(\omega) \cdot \underline{Y}(\omega)}
\end{align}
schreiben.
\\vgl. [ISV] S.180
\\\\
Wenn man nun die KKF im Frequenzbereich zeitsparend durchf"uhren will, muss man die FFT f"ur x(k) und y(k) (dabei $k \in \mathbb{N}^+_0$) der L"ange N durchf"uhren. Dabei muss man beachten, dass bei der FFT ein Linienspektrum ergibt. Die FFT beruht vor allem auch auf der Annahme, dass sich die N diskreten Werte periodisch wiederholen. Durch die IFFT von $ \underline \Psi(\omega)$ ergibt sich also die periodische KKF $\tilde{\psi}(t)$.
\\vgl. [ISV] S.135

\input{Masszahlen}
\subsection{Gauß-Regression}

%%%%%%%%%%%%%%%%%%%%%%%%%%%%%%
\section{Verfügbare Technik}
\subsection{Software}
Softwareseitig haben wir Octave benutzt. Als freie Alternative zu Matlab vereint Octave gute Performance, syntaktische Gleichheit zu Matlab, sowie kostenfreie Benutzung unter einem Dach. Im Vergleich mit Python haben wir festgestellt, dass die Geschwindigkeit aufwendiger Rechnungen, wie der Korrelation, bei Python schlechter ist. Somit haben wir uns für Ocatve entschieden. Um unseres selbstgeschriebenes Programm zu verifizieren haben wir eine Autokorrelation durchgeführt und diese mit der von Audactity berechneten AKF verglichen. Wir sind dabei zu dem Ergebnis gekommen, dass unser Programm funktioniert. Zur Aufnahme der Audiosignale haben wir ebenfalls die Software Audacity benutzt.
\subsection{Hardware}
Uns standen zwei hochwertige Kondensatormikrofone (M5 Matched Pair Compac 1/2" Cardioid Condenser Microphones von Rode) zur Verfügung. Diese Mikrofone sind für dieses Projekt besonders geeignet, da durch die Abstimmung (matched pair) nur die Unterschiede im Signal vor der Aufnahme Einfluss auf die Korrelation haben. Für die Digitalisierung der Signale stand uns ein hochwertiges Audio-USB-Interface (Scarlett 2i2 von Focusrite) zur Verfügung. Die Aufnahmen wurden in .wav gespeichert und sind somit verlustfrei. Außerdem konnten wir ein Stativ mit einer Mikrofonschiene verwenden, wodurch die Mikrofone konstanten Abstand hatten. Zur Erzeugung eines reproduzierbaren Klangsignales wurde eine portable Bluetooth-Anlage (Soundlink III von Bose) verwendet.

%%%%%%%%%%%%%%%%%%%%%%%%%%%%%%
\section{Programm}
\subsection{Aufbau des Programms}
Das Programm ist in 3 große Teile gegliedert. Dazu zählt das Einlesen der Audiodateien, die benötigte Signalverarbeitung inklusiver Berechnung der gewünschten Parameter und das Speichern der gewonnenen Werte in Form einer Excel-Datei.
\paragraph{Einlesen der Audiodaten}
Die Audiodateien liegen im WAV-Format als Stereoaufnahme vor. Zunächst wird eine Liste mit allen Dateien in einem bestimmten Ordner erstellt, damit die Dateien nacheinander eingelesen werden können. Im nächsten Schritt werden die beiden Kanäle voneinander getrennt, um diese dann in die Signalverarbeitung zu übergeben.
\paragraph{Signalverarbeitung}
Das Kernstück der Signalverarbeitung ist eine periodische Korrelationsfunktion, die den linken und rechten Kanal miteinander korreliert. Die dabei entstandene Korrelationsfunktion wird dann weiter untersucht. Als nächstes wird eine Art Einhüllende berechnet, die ein Maß für die Steilheit der Kurve ist. Wie bereits im Abschnitt 3.3 "Gauß-Regression" beschrieben, wird dann mit Hilfe der Methode der kleinsten Quadrate eine Gauß-Glocke so angepasst, dass sie den Verlauf der Hüllkurve der KKF möglichst gut abbildet. Die Parameter ripple, $\sigma$, Gleichanteil und Zeitverschiebung des Maximums aus dem Ursprung werden danach an eine Funktion übergeben, die diese Daten in einer Excel-Tabelle speichert.
\paragraph{Speicherung}
Die Speicherung der Daten erfolgt in einer Excel-Datei. Dabei wird zu erst der Dateiname des Samples und alles dazugehörigen Werte gespeichert. Außerdem wird noch ein Link zum Graphen der Korrelationsfunktion angegeben, damit man sich diese bei der Auswahl der Test-Signale anschauen kann.

\subsubsection{Programmablaufplan}
\begin{figure}[ht!]
\centering
\includegraphics[scale=0.6]{img/pap}
\caption{Programmablaufplan}
\label{figure1}
\end{figure}

\subsection{Mögliche Einstellungen}
Im Code gibt es diverse Eintellungen, die das Verhalten des Programms je nach Wunsch des Anwenders verändern. Diese sind am Beginn der main.m-Datei festgelegt und werden im folgenden beschrieben:
\begin{itemize} 
\item $path$ - Pfad zur Sammlung der WAV-Dateien
\item $excel\_path$ - Pfad unter dem Excel-Datei mit Lösungen gespeichert wird
\item $output$ - Unterscheidung, ob Ergebnisse gespeichert oder angezeigt werden
\item $calc$ - Wechsel zwischen Berechnung im Zeit- und Frequenzbereich möglich.
\item $x\_axes$ - Unterscheidung, ob Korrelation gegen Samples oder Zeit aufgetragen wird
\item $priority$ - Unterscheidung, ob angegebene Blocklänge oder Zeitdauer priorisiert wird
\item $t\_start$ - Startzeitpunkt der Korrelation
\item $t\_dur$ - Array $A1$ mit Menge an Zeitdauern die korrelierten werden sollen, Korrelation beginnt immer bei $t\_start$
\item $Ncor\_init$ - Array, mit identischer Länge zu $A1$. Gibt an wie oft korrespondierender Eintrag in $A1$ hintereinander korreliert wird. 
\item $Lcor$ - Blocklänge der Korrelation
\end{itemize}
\subsection{Probleme} 


%%%%%%%%%%%%%%%%%%%%%%%%%%%%%%
\section{Signalauswahl}
Wir haben für die Aufnahme der Signale viel verschiedene Situationen ausgesucht, um ein möglichst breites Spektrum an Raum-Effekten zu erhalten. Aufnahmeorte waren beispielsweise der Platz vor dem HSZ, die Wiese zwischen Physik- und Mathematikgebäude, sowie der Trefftzbau. Außerdem wurde in einer Wohnung gemessen, um Effekte von schallabsorbierenden Stoffen wie Teppich oder Bett zu erhalten. Soweit möglich, haben wir die natürliche Geräuschkulisse am jeweiligen Ort eingefangen. Zusätzlich dazu wurde ein definiertes Signal mittels eines Lautsprechers erzeugt, um Direktschall zu nutzen. Bei diesen Aufnahmen sollten die Effekte des Raumes am deutlichsten hervortreten.
\subsection{Beispielsignale}
\subsubsection{Signal 1 - trefftz$\_$wiese$\_$m}
\begin{figure}[ht!]
  \centering
  \includegraphics[scale=0.64]{img/trefftz_wiese_m}
  \caption{Signal 1}
  \label{figure2}
\end{figure}
\begin{figure}[ht!]
  \centering
  \includegraphics[width=0.4\textwidth]{img/wiese}
  \caption{Aufnahmesituation auf der Wiese des Trefftzbaus}
  \label{figure4}
\end{figure}
\paragraph{Aufnahmesituation} Dieses Signal wurde auf der Wiese zwischen dem Gebäude der Mathematik- und Physikfakultät aufgenommen. Dies stellt eine relativ große Freifläche mit wenigen Hindernissen mit ungehinderter Schallausbreitung dar, wobei jedoch auch die umliegenden Gebäude einen Einfluss auf das Ausbreitungsverhalten haben können. Es wurde eine zusätzliche Primärschallquelle genutzt, die während der Aufnahme stationär an einem Punkt im Raum blieb. Dieser Punkt befand sich frontal zu den Mikrofonen in einigen Metern Abstand. Durch diese Anordnung sollte zum einen ein definiertes Signal vorgegeben, aber zum anderen auch der Effekt der Umgebung erfasst werden. 
Um den Einfluss einer bewegten Schallquelle zu untersuchen, führten wir bei ähnlichem Messaufbau auch noch eine Messung durch in der wir die Primärschallquelle senkrecht zur Ausrichtung der Mikrofone bewegten. Die Ergebnisse dieser Messung sind im Anhang zu finden.
\paragraph{Signalbeschreibung} Wie in Abbildung \ref{figure2} zu sehen ist, ändern sich beide Kanäle relativ langsam. Sowohl Kanal A als auch Kanal B sind klar definiert und im Vergleich zum Rauschen relativ groß. Man erkennt jedoch bereits beim einfachen Betrachten, dass sich beide Seiten nur sehr geringfügig ähnlich sehen.
\paragraph{Beschreibung der KKF} Die Kreuzkorrelationsfunktion schwankt sehr stark über den gesamtem Zeitbereich. Deswegen ist auch die Hüllkurve stark schwankend. Da die Regression über die Hüllkurve berechnet wird, wird diese dem Signal auch nicht gerecht.
\paragraph{Auswertung der Maßzahlen}
Der ripple-Faktor von Signal 1 ist mit einem Wert von 0.2 vergleichsweise niedrig. Dass nur ein geringer Anteil der Energie des Signals in den obersten 5 Prozent der Werte vorhanden ist, lässt sich auch gut an der KKF erkennen. Es gibt nur sehr wenige Peaks. Die Aussagekräftigkeit von $\sigma$ ist hier sehr gering. Wie gut zu erkennen ist, hat die Kurve keine Ähnlichkeit mit einer Gauß-Glocke. Auch die anderen Werte sind für dieses Signal schwierig zu bewerten, da die Ähnlichkeit generell sehr klein ist.
\subsubsection{Signal 2 - trefftz$\_$fahrstuhl$\_$m}
\begin{figure}[ht!]
  \centering
  \includegraphics[scale=0.64]{img/trefftz_fahrstuhl_m}
  \caption{Signal 2}
  \label{figure3}
\end{figure}
\begin{figure}[ht!]
  \centering
  \includegraphics[width=0.4\textwidth]{img/fahrstuhl}
  \caption{Aufnahmesituation im Fahrstuhl des Trefftzbaus}
  \label{figure5}
\end{figure}

\paragraph{Aufnahemsituation} Diese Aufnahme fand im Fahrstuhl des Trefftzbaus statt. Das heißt, dass der Raum war relativ klein und ist mit dicken Glaswänden versehen. Um ein Signal zu erhalten wurde eine Primärschallquelle in Form eines hochwertigen Lautsprechers genutzt über den ein Beispielsignal wiedergegeben wurde. Da der Raum im Fahrstuhl sehr begrenzt war, ließen sich unterschiedliche Messaufbauten schwer realisieren und auch eine Bewegung der Primärschallquelle war nicht möglich. Weiterhin sei erwähnt, dass wir uns während der Messung im Fahrstuhl befanden und somit nicht nur der Raum allein vermessen wurde. Da gerade das aber eine reale Situation im Fahrstuhl ausmacht, hat dies keine negativen Auswirkungen auf die Aussagekraft des Signals. 
\paragraph{Signalbeschreibung}
In Abbildung \ref{figure3}  erkennt man sehr gut, dass sich beide Kanäle sehr schnell ändern und einen ähnlichen Verlauf haben. Lediglich die Lautstärke des Signals ist unterschiedlich. Das stört jedoch nicht für die Berechnung der Maßzahlen, da diese nicht von der Amplitude  abhängig sind. 
\paragraph{Beschreibung der KKF}
Für dieses Signal ist auch an der Kreuzkorrelationsfunktion klar zu sehen, dass es ein Maximum in der Mitte gibt. Das heißt, die Signale sind sich sehr ähnlich. Zu den Seiten nimmt die KKF langsam ab. Für solch einen Verlauf ist die Aussage der Hüllkurve und der Regression sehr gut, da diese Kreuzkorrelation gut mit einer Gauß-Kurve approxmiert werden kann.
\paragraph{Auswertung der Maßzahlen}
Der ripple-Faktor liegt mit 0.43 recht hoch, das heißt ein großer Teil der Signal-Energie liegt innerhalb des fünften Percentils. Die Aussage des Gauß-Fits ist hier wesentlich besser als bei Signal 1. Dies kann man gut in der Abbildung \ref{figure3} erkennen, denn der Verlauf der Hüllkurve kommt der Form einer Gauß-Glocke recht nah. Die Zeitverschiebung des Maximums der KKF ist minimal. Daran ist zu erkennen, dass das Signal an beiden Mikrofonen nur mit sehr kleiner Verzögerung angekommen ist, was im begrenzen Raum des Fahrstuhl auch Sinn ergibt.

\subsection{Probleme bei der Signalauswahl}
Bei der Signalauswahl ergab sich das Problem, dass man möglichst viele verschiedene Raumsituationen erfassen musste, um eine große Menge verschiedener Daten zu bekommen. Dabei war es jedoch nur schwer möglich vor Ort zu entscheiden, ob die entsprechende Aufnahme sinnvolle Ergebnisse liefert.
In den meisten Situationen war der Lautstärkepegel im Raum zu gering um 20s aufzunehmen, ohne das der Großteil der Aufnahme aus Rauschen bestand. Aus diesem Grund haben wir ein zusätzliches Signal erzeugt. Dadurch gibt es jedoch in den meisten Aufnahmen eine Primärquelle, die das Spektrum maßgeblich bestimmt.

\subsection{Fazit und Ausblick}
\subsubsection{Fazit}
Abschließend lässt sich feststellen, dass der Gauß-Fit erst ab einem ripple-Faktor von 0.3 sinnvolle Ergebnisse liefert. Bei Werten die kleiner als 0.3 sind, ist die Hüllkurve einer Gauß-Kurve zu unähnlich. Es ist jedoch festzustellen, dass die Regression der Exponentialfunktion mit kleiner werdendem ripple besser über der nach Größe sortierten Amplituden liegt.
Es ist empfehlenswert, Signale in Räumen aufzunehmen, die ausreichend klein sind, damit die Raumeffekte Auswirkungen auf das Signal haben. Sonst nimmt man größtenteils rauschen auf, welches sehr geringe Aussagen zulässt.
\subsubsection{Ausblick}
Das Ergebnis unserer Arbeit ist eine Datenbank aus aufgenommenen Audiosignalen, aber auch ein Tool zur Bewertung und Kategorisierung von Audiosignalen. Der modularisierte Code und die Einführung einiger Parameter soll eine erweiterbare Basisplattform für die Forschung am Lehrstuhl sein. Die aufgenommenen Signale sind in ihrer Diversität der Aufnahmesituationen hoffentlich von Nutzen. Wir als Gruppe haben im Rahmen dieser Projektarbeit viel im Thema Signalverarbeitung und besonders zur Implementierung von Signalverarbeitung gelernt.  

\section{Zusammenfassung}
Es wurde ein Ocatve-Skript entwickelt mit welchem sich die Kreuzkorrelationsfunktion der beiden Stereo-Kanäle einer Audioaufnahme berechnen lässt. Auf Basis der KKF wurden einige einfache Maßzahlen zur Charakterisierung der Aufnahmen entwickelt. Damit lassen sich für bestimmte Anwendungen Signale zu Testzwecken auswählen.
In Zukunft kann die Software auf bestimmte Anwendungsfälle angepasst werden, in dem neue Regressionsmodelle implementiert werden, die der gewünschen Nutzung der Signale besser gerecht werden.
